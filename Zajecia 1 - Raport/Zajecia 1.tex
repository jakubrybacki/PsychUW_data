\documentclass{beamer}
\usetheme{Berlin}
\usecolortheme{beaver}

\usepackage[T1]{fontenc}
\usepackage{hyperref}

\title{Analiza danych wtórnych – \newline
 Zajęcia 1: Jak pisać dobre raporty?}
\author{Jakub Rybacki}
\begin{document}
\begin{frame}[plain]
    \maketitle
\end{frame}

\section{Organizacyjne}
\begin{frame}{Informacje}
	Kontakt:
	\begin{itemize}
		\item \href{mailto:jakub.rybacki@uw.edu.pl}{jakub.rybacki@uw.edu.pl}
		\item Repozytorium dla materiałów: \href{https://github.com/jakubrybacki/PsychUW_Data}{\beamergotobutton{Link}} 
		\item Dyżur - w zależności od potrzeb. Prośba o kontakt mailowy. 
	\end{itemize}
	\vspace{2mm}	
	Zasady zaliczenia:
	\begin{itemize}
	\item Limit nieobecności - standardowe zasady UW
	\item Praca w trójkach - czwórkach 
	\item Projekt końcowy:
		\begin{enumerate}
			\item Część dane wtórne: krótki raport + prezentacja
			\item Część Big Data: prezentacja - wykorzystanie Power Query
		\end{enumerate}
	\end{itemize}
\end{frame}

\begin{frame}{Raport}
	Wytyczne:
	\begin{itemize}
		\item Krótka objętość 4-5 stron.  
		\item Przykładowe tematy wraz z listą przydatnych raportów w osobnym pliku. 
		\item Cel badania konsumenckie - np: 
		\begin{itemize}
			\item Czy piłka nożna będzie zmierzać w stronę Superligi? 
			\item Jak rozwija się rynek produktów luksusowych w 2023 r.?
			\item Jakie zmiany będą czekać rynek VoD w Polsce?
		\end{itemize}		
	\end{itemize}
\end{frame}

\begin{frame}{Jak pisać raporty - dobre i złe praktyki}
	\centering
	Wszystko co tu będziemy omawiać ma charakter subiektywny.
	
	\vspace{5 mm}
	Uznajemy to za porady.
	
	\vspace{5 mm}
	Przykłady dotyczyć będą badań ekonomicznych - to mój rynek.
\end{frame}

\section{Treść}
\begin{frame}{Najczęstsze problemy}
	\begin{columns}[t]
		\column{0.5\textwidth}
		\centering
		Ściana tekstu:
		\includegraphics[keepaspectratio, width=0.4\paperwidth]{./Wykresy/CA_MacroMapa.jpg}
		\tiny
		Źródło: \href{https://static.credit-agricole.pl/asset/m/a/k/makromapa-20220214_22757.pdf}{Credit Agricole}  
		
		\column{0.5\textwidth}
		\centering
		Nadmiar myśli + żargon:
		\includegraphics[keepaspectratio, width=0.4\paperwidth]{./Wykresy/ING_Natlok.jpg}	
		\tiny
		Źródło: ING Bank Śląski
	\end{columns}
\end{frame}

\begin{frame}{Dobre praktyki}
	\begin{columns}[t]
		\column{0.5\textwidth}
		Tytuł - 1 linijka		
		\\
		\vspace{5mm}
		Paragrafy:
		\begin{enumerate} 
			\item Objętość: Najlepiej 3-5 linijek tekstu.
			\item Pierwsza linijka zdradza przesłanie paragrafu (pogrubiona).
			\item Związek logiczny między kolejnymi rozpoczęciami
			\item Pogrubione fragmenty dają pełny obraz raportu
		\end{enumerate}		
	
		\column{0.5\textwidth}
		\centering
		Konstrukcja raportu:
		\includegraphics[keepaspectratio, width=0.4\paperwidth]{./Wykresy/PIE_raport.jpg}
		\tiny
		Źródło: \href{https://pie.net.pl/wp-content/uploads/2021/12/PIE-Przeglad_gospodarczy_2-2021.pdf}{Polski Instytut Ekonomiczny}  
	\end{columns}
\end{frame}

\begin{frame}{Dobre praktyki - cd.}
	\begin{columns}[t]
		\column{0.5\textwidth}
		Kluczowe liczby:
		\begin{itemize}
			\item Ułatw odbiorcy dotarcie do głównych myśli
			\item Ty decydujesz co przeczyta.
			\item Sekcja kluczowych liczb może zaczynać raport.
		\end{itemize}		
		\vspace{5mm}

		Pytanie: Które informacje są najważniejsze dla sekcji raportu?

		\column{0.5\textwidth}
		
		\centering
		Przykład - kluczowe myśli
		\includegraphics[keepaspectratio, width=0.4\paperwidth]{./Wykresy/ING_Think.jpg}
		\tiny
		Źródło: \href{https://think.ing.com/bundles/ing-monthly-the-masks-and-the-gloves-are-coming-off-economics/}{ING Think}  
	\end{columns}
\end{frame}

\begin{frame}{Najczęściej czytane elementy raportów}
	\begin{enumerate}
		\item Komunikat prasowy
		\begin{itemize}
			\item Zwykle sporządza biuro prasowe - bez wiedzy eksperckiej.
			\item Zaniedbanie - uderza w czytelnictwo
		\end{itemize}
		
		\vspace{5mm}
		\item Synteza (ang. \emph{Executive Summary})
		\begin{itemize}
			\item Pełna kontrola autora.
			\item Mało liczb.
		\end{itemize}
		
		\vspace{5mm}
		\item Wykresy / Tabele
		\begin{itemize}
			\item Uwaga na tytuł wykresu.
			\item Cel: Treść sama się objaśnia. 
		\end{itemize}			
	\end{enumerate}
\end{frame}

\section{Styl}
\begin{frame}{Problem - zamiłowanie do retoryki}
	Krótki film dla zobrazowania: \href{https://www.youtube.com/watch?v=MJLCcQPgye0}{\beamergotobutton{Link}}
	
	\vspace{5mm}
	Raporty analityczne:
	\begin{itemize}	
		\item Waszym klientem jest Stefan (ten 2-gi z filmu)!
		\item Stefan ceni prosty i informatywny język
		\item Cel: Raport ma być przyjemny, nawet następnego dnia po imprezie. 
	\end{itemize}
\end{frame}

\begin{frame}{Język biznesowy inny niż akademicki}
	\begin{enumerate}
		\item Strona czynna - forma aktywna:
			\begin{itemize}
				\item TAK: Rząd wprowadził błędne reformy.
				\item NIE: Wprowadzono błędne reformy. 
			\end{itemize}
		\vspace{5mm}
		\item Formułowanie opinii:
			\begin{itemize}
				\item Tekst biznesowy - główna wartość.
				\item Tekst akademicki - niepożądane.
			\end{itemize}	
		
		\vspace{5mm}
		\item Punkty wspólne, które najczęściej się edytuje:
			\begin{itemize}
				\item Konstrukcje gramatyczne ala mistrz Yoda. 
				\item Powtórzenia
				\item Styl: unikamy przeczeń.
		\end{itemize}	
		
	\end{enumerate}	

\end{frame}

\section{Wykresy}
\begin{frame}{Problem - Nieczytelne wykresy}
	\begin{columns}[t]
		\column{0.5\textwidth}
		Najczęstsze mankamenty:
		\begin{itemize}
			\item Hermetyczne nazewnictwo (prawa strona)
			\item Niejasne związki między seriami.
			\item Dużo niepotrzebnych informacji np. powtarzane daty
		\end{itemize}		

		\column{0.5\textwidth}
		\centering
		Niefortunny wykres
		\includegraphics[keepaspectratio, width=0.4\paperwidth]{./Wykresy/Millenium_Wykres.jpg}
		\tiny
		Źródło: \href{https://www.bankmillennium.pl/documents/10184/29721383/Makro_i_Rynek_2022.pdf/}{Bank Millenium}  
	\end{columns}
\end{frame}

\begin{frame}{Problem - Nieczytelne wykresy}
	\begin{columns}[t]
		\column{0.5\textwidth}
		\centering
		Słabo
		\includegraphics[keepaspectratio, width=0.5\paperwidth]{./Wykresy/NBP_WIBOR3M.jpg}
		\tiny
		Źródło: \href{https://www.nbp.pl/publikacje/materialy_i_studia/ms337.pdf}{NBP}  
		
		\column{0.5\textwidth}
		\centering
		Dobrze
		\includegraphics[keepaspectratio, width=0.4\paperwidth]{./Wykresy/CeNEA_WIBOR3M.jpg}
		\tiny
		Źródło: \href{https://twitter.com/MichalMyck/status/1494312342984826890?s=20&t=NXvtdJQDxmTJo1QYSe-EhQ}{@MichalMyck}  
	\end{columns}
\end{frame}


\begin{frame}{Trudne tematy - Ważna jest strategia!}
	Za ładnym formatowaniem wykresów można ukryć mankamenty merytoryczne!
	\vspace{5mm}
	
	Ciekawe praktyki - dla chwili rozluźnienia:
	\begin{enumerate}
		\item Incydent? \href{https://www.youtube.com/watch?v=0OvL0eBS-GQ}{\beamergotobutton{Link}}
		\item Nie sądze!
		\href{https://www.youtube.com/watch?v=MVpddez7DLA}{\beamergotobutton{Link}}
	\end{enumerate}
	
\end{frame}


\section{Prezentacja wyników}
\begin{frame}{Dashboard}
	Kiedyś analizy bazowały na notatkach. 
	\vspace{5mm}	
	Dzisiaj świat jest dużo bogatszy!
	\vspace{5mm}
	
	Dane przedstawiane są w formie interaktywnej: 
	\href{	https://www.bruegel.org/dataset/national-policies-shield-consumers-rising-energy-prices}{\beamergotobutton{Link}}
\end{frame}

\begin{frame}{Social Media}
	Ważna zasada pracy w korporacji - czyń swoją pracę widoczną.

	\vspace{5mm}	 
	Zasięg / twórczość Social media pomaga w tym. To też zasób podczas rekrutacji. 
	
	\vspace{5mm}	
	Kanały:
	\begin{enumerate}
		\item Twitter - komunikacja z dziennikarzami. 
		\item LinkedIn 
	\end{enumerate}
	
\end{frame}


\section{Word}
\begin{frame}{Umiejętności techniczne:}
	
	\begin{enumerate}
		\item Formatowanie stylu: \href{	https://support.microsoft.com/en-us/office/customize-or-create-new-styles-d38d6e47-f6fc-48eb-a607-1eb120dec563
		}{\beamergotobutton{Link}} 
		\item Dynamiczne tytułowanie wykresów: \href{https://support.microsoft.com/en-us/office/add-format-or-delete-captions-in-word-82fa82a4-f0f3-438f-a422-34bb5cef9c81}{\beamergotobutton{Link}}
		\item Tryb śledzenia zmian: \href{https://support.microsoft.com/en-us/office/track-changes-in-word-197ba630-0f5f-4a8e-9a77-3712475e806a}{\beamergotobutton{Link}}
		\item Wtyczka Mendeley Cite. 
	\end{enumerate}
	\vspace{5mm}
			
\end{frame}


\end{document}

